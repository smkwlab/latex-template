\documentclass[dvipdfmx,uplatex,a4paper,10pt]{jsarticle}

%% 基本パッケージ
\usepackage{graphicx}                    % 図表挿入
\usepackage{amsmath,amssymb}            % 数式
\usepackage{url}                        % URL表示
\usepackage{enumitem}                   % リスト調整
\usepackage{textcomp}                   % 追加記号サポート

%% 日本語フォント設定
\usepackage[deluxe]{otf}
\usepackage[noalphabet,unicode,haranoaji]{pxchfon}

%% レイアウト設定
\usepackage[top=25mm,bottom=25mm,left=25mm,right=25mm]{geometry}
\linespread{1.2}                        % 行間調整

%% リスト設定
\renewcommand{\labelitemi}{$\bullet$}
\renewcommand{\labelitemii}{$\circ$}

%% ハイパーリンク設定(シンプル)
\usepackage[hidelinks]{hyperref}

%% 文書情報
\title{LaTeX文書テンプレート}
\author{著者名}
\date{\today}

\begin{document}
\maketitle

% 目次(簡素化)
\tableofcontents
\newpage

\section{はじめに}

これは汎用的な日本語LaTeX文書のテンプレートです。研究ノート、レポート、実験記録など様々な用途に活用できます。

\subsection{このテンプレートの特徴}

\begin{itemize}
\item シンプルな構造で使いやすい
\item 日本語に最適化されたレイアウト
\item 基本的な機能を網羅
\item カスタマイズが容易
\end{itemize}

\section{基本的な使い方}

\subsection{文書の構成}

文書は\texttt{section}と\texttt{subsection}で構成します。必要に応じて\texttt{subsubsection}も使用できます。

\subsection{数式の記述}

数式は以下のように記述できます。

インライン数式: $E = mc^2$

独立した数式:
\begin{equation}
(x - a)^2 + (y - b)^2 = r^2
\label{eq:circle}
\end{equation}

式~\ref{eq:circle}は円の方程式です。

\subsection{図表の挿入}

図や表は\texttt{graphicx}パッケージを使用して挿入できます。

% 図の例(コメントアウト)
% \begin{figure}[htbp]
%   \centering
%   \includegraphics[width=0.8\textwidth]{example.png}
%   \caption{図の例}
%   \label{fig:example}
% \end{figure}

\section{応用例}

\subsection{リストの活用}

番号なしリスト:
\begin{itemize}
\item 項目1
\item 項目2
  \begin{itemize}
  \item サブ項目A
  \item サブ項目B
  \end{itemize}
\item 項目3
\end{itemize}

番号付きリスト:
\begin{enumerate}
\item 手順1
\item 手順2
\item 手順3
\end{enumerate}

\subsection{引用と参考文献}

重要な内容は以下のように引用できます:

\begin{quote}
ここに引用文を記述します。長い引用や重要な文章を強調する際に使用します。
\end{quote}

\section{まとめ}

このテンプレートを基に、目的に応じてカスタマイズしてください。LaTeX文書作成の効率向上に役立てていただければ幸いです。

\end{document}